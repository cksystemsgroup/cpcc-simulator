%% LaTeX Beamer presentation template (requires beamer package)
%% see http://latex-beamer.sourceforge.net/
%% idea contributed by H. Turgut Uyar
%% template based on a template by Till Tantau
%% this template is still evolving - it might differ in future releases!

\documentclass{beamer}
\usepackage[utf8]{inputenc}
\usepackage{etex}

\mode<presentation>
{
\usetheme{Boadilla}
%\usetheme{Dresden}
%\usetheme{Madrid}
%\usetheme{Singapore}

\usecolortheme{wolverine}
%\usecolortheme{crane}
%\usecolortheme{dove}
%\usecolortheme{seagull}
%\usecolortheme{seahorse}
%\usecolortheme{rose}

\setbeamerfont{title}{shape=\itshape,family=\rmfamily}
%\setbeamercolor{title}{fg=red!80!black}
%\setbeamercolor{title}{fg=red!80!black,bg=red!20!white}

\usefonttheme{serif}
%\usefonttheme{structuresmallcapsserif}

\setbeamercovered{transparent}
}


\usepackage[german]{babel}
\usepackage[german]{babelbib}

% font definitions, try \usepackage{ae} instead of the following
% three lines if you don't like this look
\usepackage{mathptmx}
\usepackage[scaled=.90]{helvet}
\usepackage{courier}


\usepackage[T1]{fontenc}
\usepackage{pictex}
\usepackage{dsfont}
\usepackage{ulem}


\title{ESE Project}

%\subtitle{}

% - Use the \inst{?} command only if the authors have different
%   affiliation.
%\author{F.~Author\inst{1} \and S.~Another\inst{2}}
\author{M. Kleber, C. Krainer, A. Schr\"ock, B. Zechmeister}

% - Use the \inst command only if there are several affiliations.
% - Keep it simple, no one is interested in your street address.
\institute[University of Salzburg]
{
% \inst{1}%
Department of Computer Sciences\\
  University of Salzburg, Austria
% \and
% \inst{2}%
% Department of Theoretical Philosophy\\
% Univ of E
}

\date{\today}


% This is only inserted into the PDF information catalog. Can be left
% out.
\subject{Talks}



% If you have a file called "university-logo-filename.xxx", where xxx
% is a graphic format that can be processed by latex or pdflatex,
% resp., then you can add a logo as follows:

% \pgfdeclareimage[height=0.5cm]{university-logo}{university-logo-filename}
% \logo{\pgfuseimage{university-logo}}



% Delete this, if you do not want the table of contents to pop up at
% the beginning of each subsection:
\AtBeginSubsection[]
{
\begin{frame}<beamer>
\frametitle{Outline}
\tableofcontents[currentsection,currentsubsection]
\end{frame}
}

% If you wish to uncover everything in a step-wise fashion, uncomment
% the following command:

%\beamerdefaultoverlayspecification{<+->}

\begin{document}

\begin{frame}
\titlepage
\end{frame}

\begin{frame}
\frametitle{Inhalt}
\tableofcontents
% You might wish to add the option [pausesections]
\end{frame}


\section{Introduction}
%\subsection[Short First Subsection Name]{First Subsection Name}

\begin{frame}
\frametitle{Introduction}
%\framesubtitle{xxx}

Test $\displaystyle{x = (x_n)_{n\in\mathds{N}_0}}$

Test
\begin{equation}
	x : \mathds{N}_0 \rightarrow \mathds{C}
\end{equation}

\pause

Test
\begin{equation}
	\mathcal{Z}\{x\} =
	X(z) := \sum_{n=0}^\infty x_n\cdot z^{-n} ~, \qquad z\in \mathds{C}
\end{equation}

\pause
\vspace{0.5cm}
$\displaystyle{X(z)}$ ist f\"ur die $\displaystyle{z\in\mathds{C}}$ definiert, f\"ur die die
unendliche Reihe konvergiert.

%$\displaystyle{\sum_{k=0}^\infty x_k\cdot z^{-k}}$ konvergent ist.

\end{frame}


\section{Implementation}

\begin{frame}
\frametitle{Implementation}
%\framesubtitle{xxx}

Test

\end{frame}




\section{Results}
%\subsection[Short First Subsection Name]{First Subsection Name}

\begin{frame}
\frametitle{Results}
%\framesubtitle{xxx}

Test

\end{frame}










\section{Questions and Answers}

\begin{frame}
	\frametitle<presentation>{Questions \& Answers}
	\framesubtitle{~}
	\begin{center}
		\fontsize{64}{64}\selectfont
		\begin{tabular}{ccc}
			Q &&\\
			& ~~\& & \\
			&&~~ A
		\end{tabular}
	\end{center}
\end{frame}



\end{document}
