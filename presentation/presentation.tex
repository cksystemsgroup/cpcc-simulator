%% LaTeX Beamer presentation template (requires beamer package)
%% see http://latex-beamer.sourceforge.net/
%% idea contributed by H. Turgut Uyar
%% template based on a template by Till Tantau
%% this template is still evolving - it might differ in future releases!

\documentclass{beamer}
\usepackage[utf8]{inputenc}
\usepackage{etex}

\mode<presentation>
{
\usetheme{Boadilla}
%\usetheme{Dresden}
%\usetheme{Madrid}
%\usetheme{Singapore}

\usecolortheme{wolverine}
%\usecolortheme{crane}
%\usecolortheme{dove}
%\usecolortheme{seagull}
%\usecolortheme{seahorse}
%\usecolortheme{rose}

\setbeamerfont{title}{shape=\itshape,family=\rmfamily}
%\setbeamercolor{title}{fg=red!80!black}
%\setbeamercolor{title}{fg=red!80!black,bg=red!20!white}

\usefonttheme{serif}
%\usefonttheme{structuresmallcapsserif}

\setbeamercovered{transparent}
}


\usepackage[german]{babel}
\usepackage[german]{babelbib}

% font definitions, try \usepackage{ae} instead of the following
% three lines if you don't like this look
\usepackage{mathptmx}
\usepackage[scaled=.90]{helvet}
\usepackage{courier}


\usepackage[T1]{fontenc}
\usepackage{pictex}
\usepackage{dsfont}
\usepackage{ulem}


\title{ESE Project}

%\subtitle{}

% - Use the \inst{?} command only if the authors have different
%   affiliation.
%\author{F.~Author\inst{1} \and S.~Another\inst{2}}
\author{M. Kleber, C. Krainer, A. Schr\"ocker, B. Zechmeister}

% - Use the \inst command only if there are several affiliations.
% - Keep it simple, no one is interested in your street address.
\institute[University of Salzburg]
{
% \inst{1}%
Department of Computer Sciences\\
  University of Salzburg, Austria
% \and
% \inst{2}%
% Department of Theoretical Philosophy\\
% Univ of E
}

\date{\today}


% This is only inserted into the PDF information catalog. Can be left
% out.
\subject{Talks}



% If you have a file called "university-logo-filename.xxx", where xxx
% is a graphic format that can be processed by latex or pdflatex,
% resp., then you can add a logo as follows:

% \pgfdeclareimage[height=0.5cm]{university-logo}{university-logo-filename}
% \logo{\pgfuseimage{university-logo}}



% Delete this, if you do not want the table of contents to pop up at
% the beginning of each subsection:
\AtBeginSubsection[]
{
\begin{frame}<beamer>
\frametitle{Outline}
\tableofcontents[currentsection,currentsubsection]
\end{frame}
}

% If you wish to uncover everything in a step-wise fashion, uncomment
% the following command:

%\beamerdefaultoverlayspecification{<+->}

\begin{document}

\begin{frame}
\titlepage
\end{frame}

\begin{frame}
\frametitle{Inhalt}
\tableofcontents
% You might wish to add the option [pausesections]
\end{frame}


% - Einleitung
%   - Aufgabenstellung
%   - Systemarchitektur: Pilot, Engine, Mapper, GMView, Planner, ...
%   - Datenaustausch mit HTTP(S)
\section{Introduction}
%\subsection[Short First Subsection Name]{First Subsection Name}

\begin{frame}\frametitle{Target} %\framesubtitle{xxx}
\begin{itemize}
\item \ldots
\end{itemize} 
\end{frame}



% - Real Vehicles
%   - Flugplan / VCL
%   - Simulation
%   - Sensoren
\section{Real Vehicles}

\begin{frame}\frametitle{Flight Plan} %\framesubtitle{xxx}
\begin{itemize}
\item real vehicles follow strict flight plans
\item \ldots
\end{itemize} 
\end{frame}

% - Vehicle Virtualisierung
%   - Prinzipielle Funktion eines VV-Programms
%   - Status / Serialisierung / Migration
%   - VV-Language: Command = Point + Actions
%   - Scanner, Parser
\section{Vehicle Virtualization}

\begin{frame}\frametitle{Virtual Vehicle Program} %\framesubtitle{xxx}
\begin{itemize}
\item ability to suspend
\item state is serialized
\item information is persisted to file
\item migration can be performed
\item virtual vehicle can resume
\end{itemize} 
\end{frame}

\begin{frame}\frametitle{Virtual Vehicle Language} %\framesubtitle{xxx}
\begin{itemize}
\item list of commands
\item command consits of a point and a list of actions
\item point contains latitude, longitude, altitude
\item specification of tolerance
\end{itemize} 
\end{frame}

\begin{frame}\frametitle{Virtual Vehicle Sample Program} %\framesubtitle{xxx}
\texttt{\\
Point 47.82201946 13.04082647 1.00 tolerance 12.3\\
Picture \\
Temperature\\
\\
Point 47.82203026 13.04084659 25.00 tolerance 100 \\
Temperature\\
\\
Point 47.82211311 13.04076076 30.00 tolerance 1.2\\
Picture} 
\end{frame}

\begin{frame}\frametitle{Scanner \& Parser} %\framesubtitle{xxx}
\begin{itemize}
\item \ldots
\end{itemize} 
\end{frame}

% - Mapping
%   - Registrierung der Engines
%   - Algorithmen: Random, Simple
%   - Ansto� der Migration
\section{Mapping}

\begin{frame}\frametitle{Engine Registration} %\framesubtitle{xxx}
\begin{itemize}
\item \ldots
\end{itemize} 
\end{frame}

% - Erk�rung User-Interface:
%   - GM-View
\section{User Interface}

\begin{frame}\frametitle{Google Maps Viewer} %\framesubtitle{xxx}
\begin{itemize}
\item \ldots
\end{itemize} 
\end{frame}


\section{Live Demonstration}
% - Demo 1: "VV sammelt Daten"
%   - Setup: 1xRV, 1xVV, 2xVV-Points
%   - RV fliegt die VV-Points ab
%   - VV sammelt die Daten ein
%\subsection{Demo 1 - Data Collection}

\begin{frame}\frametitle{Data Collection} %\framesubtitle{xxx}
\begin{itemize}
\item \ldots
\end{itemize} 
\end{frame}

% - Demo 2: "VV wird migriert"
%   - Setup: 2xRV, 1xVV, 2xVV-Points
%   - Je ein RV fliegt einen VV-Point an
%   - VV sammelt am RV1 die ersten VV-Point Daten
%     ein, wird migriert und macht am RV2 weiter.
%\subsection{Demo 2 - Migration}

\begin{frame}\frametitle{Virtual Vehicle Migration} %\framesubtitle{xxx}
\begin{itemize}
\item \ldots
\end{itemize} 
\end{frame}

% - Demo 3: "RVs haben unterschiedliche Sensoren"
%   - Setup: 3xRV, 1xVV, 1xVV-Point
%   - Alle RVs fliegen nacheinander einen VV-Point an
%   - VV sammelt am RV1 die ersten VV-Point Daten
%     ein, wird migriert auf RV2, sammelt weiter Daten
%     ein, wird migriert auf RV3 und sammelt die
%     restlichen Daten ein.
%\subsection{Demo 3 - Sensor Variety}

\begin{frame}\frametitle{Real Vehicles with different Sensors} %\framesubtitle{xxx}
\begin{itemize}
\item \ldots
\end{itemize} 
\end{frame}


% - Demo 4: "All in one."
%   - Setup: 3xRV, 3xVV, 9xVV-Point
%   - ...
%\subsection{Demo 4 - All in One}

\begin{frame}\frametitle{All in One} %\framesubtitle{xxx}
\begin{itemize}
\item \ldots
\end{itemize} 
\end{frame}

% - Future Work
%   - Aufw�ndigere Mapping-Algorithmen
%   - Optimierung Netzwerkverkehr
%   - Video-Sensor
%   - Geo-Location: Vergleich RV-Position zu VV-Sollwert
%   - RV-Flugpl�ne aus VV-Programmen zusammenstellen
\section{Future Work}

\begin{frame}\frametitle{Future Work} %\framesubtitle{xxx}
\begin{itemize}
\item more sophisticated mapping algorithms
\item optimized network traffic
\item video sensor support
\item geo-location
\item flight plan generation
\end{itemize} 
\end{frame}
 

% - Q & A Session
\section{Questions and Answers}

\begin{frame}
	\frametitle<presentation>{Questions \& Answers}
	\framesubtitle{~}
	\begin{center}
		\fontsize{64}{64}\selectfont
		\begin{tabular}{ccc}
			Q &&\\
			& ~~\& & \\
			&&~~ A
		\end{tabular}
	\end{center}
\end{frame}

% Sample Frame:
% \begin{frame}
% \frametitle{Introduction}
% %\framesubtitle{xxx}
% 
% Test $\displaystyle{x = (x_n)_{n\in\mathds{N}_0}}$
% 
% Test
% \begin{equation}
% 	x : \mathds{N}_0 \rightarrow \mathds{C}
% \end{equation}
% 
% \pause
% 
% Test
% \begin{equation}
% 	\mathcal{Z}\{x\} =
% 	X(z) := \sum_{n=0}^\infty x_n\cdot z^{-n} ~, \qquad z\in \mathds{C}
% \end{equation}
% 
% \pause
% \vspace{0.5cm}
% $\displaystyle{X(z)}$ ist f\"ur die $\displaystyle{z\in\mathds{C}}$ definiert, f\"ur die die
% unendliche Reihe konvergiert.
% 
% %$\displaystyle{\sum_{k=0}^\infty x_k\cdot z^{-k}}$ konvergent ist.
% 
% \end{frame}

\end{document}
