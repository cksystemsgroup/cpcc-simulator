%%
%% Conclusion
%%

\chapter{Conclusion}

This work has presented an implementation of a simulation system that demonstrates information-acquisition-as-a-service
of mobile sensor networks for \ac{CPCC} as proposed in \cite{HotCloud10}.
%
This chapter concludes the paper by summarizing the current situation, and providing suggestions for future
enhancements.

Our goal was to implement a flexible and scalable simulation system. We applied 
the Java programming language and standard Internet technologies like web services to meet this challenge.  

The implemented system currently allows the simulation of helicopter fleets of several dozens of vehicles
and supports the simulation of sensors like GPS receivers and photo cameras.
To simulate air-pressure sensors, temperature sensors, etc.~the system utilizes random number generators, which
deliver values in a defined range and precision.
%
Support for hardware-in-the-loop testing is available for flight control systems and helicopter plants.
%

Simulated helicopters follow strict flight plans, but do not access the onboard sensors for data collection.
It is a virtual abstraction of autonomous vehicles, \acfp{VV} for short, that gathers data.
One helicopter is able to carry several \acp{VV}. To complete their missions, \acp{VV} may migrate between helicopters.


Future works could cover the following topics:
\begin{itemize}
  \item The implemented mapping algorithm considers only the current flight plan segment for migration decisions.
  	Future implementations should include all helicopter set course segments.
   
   \item Flight plans for helicopters should be derived from \ac{VV} mission requirements. 
   
   \item Although the implementation is able to simulate dozens of helicopters, the network traffic between
   	helicopters and ground station needs optimization to achieve higher scalability.
   
   \item More advanced camera sensors may allow for directing the sensors towards defined targets, which requires
   	extending the \ac{VV} programming language.
   
   \item In \ac{VV} missions action points define where to capture sensor values. Video cameras  
   	and other streaming sources need new \ac{VV} programming language commands to trigger recordings.

\end{itemize}

