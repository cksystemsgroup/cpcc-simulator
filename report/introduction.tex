%%
%% Introduction
%%

\chapter{Introduction}

Our goal was to build a simulation system that demonstrates information-acquisition-as-a-service
of mobile sensor networks for \ac{CPCC} as proposed in \cite{HotCloud10}.
Based on the JNavigator project \cite{CKrainer2009} our implementation provides
\begin{itemize}
	\item the simulation of physical helicopter swarms,
	\item the simulation of sensors,
	\item the virtual abstraction of autonomous vehicles (virtual vehicles), and
	\item the migration of virtual vehicles among flying physical helicopters (real vehicles).
\end{itemize} 

We consider this project as a first step into the domain of information-acquisition-as-a-service
and therefore allow the following limitations:
\begin{itemize}
	\item Real vehicles follow strict flight plans.
	\item There are no network bandwith limits.
	\item There are no processing power limits.
\end{itemize} 

% leave the following items to future work:

In this project,
\begin{itemize}
	\item we apply \ac{HTTP} as protocol for sensor abstraction and data exchange,
	\item we use Java as programming language,
	\item we implement the software as web applications, and
	\item we utilize Apache Tomcat as web server and servlet container.
\end{itemize} 

This document describes the highlights of the implemented software.
Chapter 2 reveals the implementation details,
chapter 3 describes the project results,
and chapter 4 summarises this paper by depicting proposals for future enhancements.
