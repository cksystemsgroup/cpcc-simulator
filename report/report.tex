
\documentclass[a4paper,12pt]{report}
%\usepackage[x11names, rgb]{xcolor}
\usepackage[utf8]{inputenc}

\usepackage{makeidx}
\usepackage{longtable}
\usepackage{graphicx}
\usepackage[english]{babel}
\usepackage{babelbib}
\usepackage{verbatim}
\usepackage{verbatimfiles}
\usepackage{alltt}
\usepackage[printonlyused]{acronym}
\usepackage[pdftex,plainpages=false,pdfpagelabels,hyperfootnotes=false]{hyperref}
\usepackage{setspace}
\lefthyphenmin=2
\righthyphenmin=3
\usepackage{afterpage}
\usepackage{csplus}
\usepackage{amsmath}
\usepackage{latexsym}
% \usepackage{thesis}
%\usepackage{pictex}
%\usepackage{specialwords}
\usepackage[binary,squaren]{SIunits}
%\usepackage{hhline}
\usepackage{url}
% \usepackage{dsfont}
% \usepackage{MnSymbol}
\usepackage{listings}
\usepackage{ragbag}
% \usepackage{enumerate}
\raggedbottom

\csplusTitle{ESE CPCC Project}
\csplusAuthor{M. Kleber, C. Krainer, A. Schr\"ocker, B. Zechmeister}
\csplusDate{\today}
\csplusReport{WS 2011/2012}
%\csplusURL{http://www.cs.uni-salzburg.at}
\csplusURL{}
\raggedbottom
\makeindex
\makeglossary


\begin{document}

\setcounter{tocdepth}{3}
\csplusMaketitle
%\onehalfspacing
\begin{abstract}
For the hands-on part of the Embedded Software Engineering Course, Winter 2011/2012 of Prof.~C.~Kirsch
(Department of Computer Sciences, University of Salzburg) we decided to build a simulation system that
demonstrates information-acquisition-as-a-service of mobile sensor networks for \acl{CPCC}.

This paper presents the main concepts of our project.
First, this document describes the goals of this project, followed by implementation characteristics, limitations, and
applied technologies.
%
Second, this document outlines the system, reveals sensor simulation, explains configuration parameters,
describes vehicle virtualization, and pictures cyber-mobility.
%
Third, it elaborates on the class demonstrations, which exemplify data collection, vehicle migration, and varying sensor
equipment.
%
Finally, this paper summarizes the implemented system and depicts proposals for future enhancements.
\end{abstract}


\tableofcontents
\setcounter{tocdepth}{3}
\clearpage
\phantomsection
\newpage

\parindent 0pt

%%
%% Introduction
%%

\chapter{Introduction}

Our goal was to build a simulation system that demonstrates information-acquisition-as-a-service
of mobile sensor networks for \ac{CPCC} as proposed in \cite{HotCloud10}.
Based on the JNavigator project \cite{CKrainer2009} our implementation provides
\begin{itemize}
	\item the simulation of physical helicopter swarms,
	\item the simulation of sensors,
	\item the virtual abstraction of autonomous vehicles (virtual vehicles), and
	\item the migration of virtual vehicles among flying physical helicopters (real vehicles).
\end{itemize} 

We consider this project as a first step into the domain of information-acquisition-as-a-service
and therefore allow the following limitations:
\begin{itemize}
	\item Real vehicles follow strict flight plans.
	\item There are no network bandwidth limits.
	\item There are no processing power limits.
\end{itemize} 

% leave the following items to future work:

In this project,
\begin{itemize}
	\item we apply \acs{HTTP} \cite{RFC_2616} as protocol for sensor abstraction and data exchange,
	\item we use Java as programming language,
	\item we implement the software as web applications, and
	\item we utilize Apache Tomcat \cite{TOMCAT2012} as web server and servlet container.
\end{itemize} 

This document describes the highlights of the implemented software.
Chapter 2 reveals the implementation details,
chapter 3 describes the project results,
and chapter 4 summarizes this paper and depicts proposals for future enhancements.


%%
%% Implementation
%%

\chapter{Implementation}

Thist chapter \ldots

\section{System Overview}

Figure~\ref{fig:SystemOverview} presents an overview of the complete system containing
one simulated \ac{RV} and the ground station.

The simulated \ac{RV} mainly comprises an Apache Tomcat web container, which executes a
Pilot web pplication and an Engine web application.
%
The Pilot web application consists of model helicopter plant simulator, that is, the MockJAviator,
a flight control system, an auto pilot, and a sensor simulator.
The MockJAviator emulates the helicopter's flight dynamics and \ac{IMU},
the flight control system operates attitude and altitude of the simulated vehicle, 
and the autopilot stirs the simulated vehicle along a \ac{VCL} script defined trajectory.
The sensor simulator supports GPS reveivers, belly mounted cameras, thermometers, barometers,
and sonar sensors.
\begin{figure}[h]
	\begin{center}
		{\includegraphics[width=10cm]{SystemOverview.pdf}}
	\end{center}
	\caption{System Overview.\label{fig:SystemOverview}}
\end{figure}

The ground station executes an Apache Tomcat web container, which runs a Google Maps Viewer
web application, a Engine web application, and a Mapper web application.
%
The Google Maps Viewer web application allows an operator to supervise ongoing missions.
%
The ground station Engine web application provides a \ac{VV RTE} for uploading and downloading \acp{VV}. 
%
Registry and mapping algorithm are the main components of the Mapper web application.
The mapping algorithm assigns \acp{VV} to \acp{RV}, based on fight plans and available sensors.
%
Each Engine web application registers with the Mapper's registry.


\section{Sensor Simulation}
Figure~\ref{fig:SensorSimulation} visualizes the simulation of sensors.
%
The current implementation supports belly mounted cameras, random number generators for emulating thermometers and
barometers, GPS position sensors, and sonar sensors.

 
\begin{figure}[h]
	\begin{center}
		{\includegraphics[width=11.6cm]{SensorSimulation-3.pdf}}
	\end{center}
	\caption{Sensor Simulation.\label{fig:SensorSimulation}}
\end{figure}


The belly mounted camera delivers a Google Maps satellite image indicating the position of the simulated vehicle by a
visual marker, exemplified by Figure~\ref{fig:BellyMountedCamera}.
To achieve this, the Pilot web application supplies a JavaScript enhanced HTML page.
\begin{figure}[h]
	\begin{center}
		{\includegraphics[width=6cm]{bmc-photo-cosy.png}}
	\end{center}
	\caption{A photo captured by the belly mounted camera.\label{fig:BellyMountedCamera}}
\end{figure}

Once the Firefox web browser selects this page, the embedded JavaScript program periodically polls the Pilot web
application for the vehicle's current position. After that, the JavaScript program slides the center of the
displayed satellite view to this position and repositions the marker.
To allow several belly mounted cameras beeing simulated on the same machine, the utilized web browsers use \ac{Xvfb}
devices as output screens.
%
Whenever the belly mounted camera needs to deliver a photo, it applies the \texttt{xwd} utility to take a snapshot
of the corresponding \ac{Xvfb} device, converts this snapshot to an image in PNG format by using program
\texttt{convert}, and delivers it via \ac{HTTP}.

Thermometer and air pressure sensors apply random number generators to simulate values.
The position sensor queries the GPS receiver simulator for the current position of the vehicle, and the sonar
sensor reads the current altitude over ground from the instantiated MockJAviator.

\section{Configuration}
This section covers the configuration of the web applications Pilot, Engine, Mapper, and GM-Viewer.

\subsection{Pilot Web Application}
The configuration of the Pilot web application consists of three parts. The first part is the configuration of the
simulated model helicopter hardware, the second part is the configuration of simulated sensors, and the third
part is the assigned \ac{VCL} script.

Listing~\ref{lst:virtHWconfig} depicts a fully simulated \ac{RV}. With the property \texttt{plant.simulated} equal
to \texttt{true}, the Pilot web application simulates the model helicopter's flight dynamics of the physical body
including the \ac{IMU}.
The simulated model helicopter of type \texttt{MockJAviator} 
awaits instructions via \acs{UDP} on address \texttt{localhost} port \texttt{9011} from a controller that
operates attitude and altitude. 
\lstset{tabsize=3,language=Tex}
\begin{lstlisting}[caption={Virtual Hardware Configuration Example},mathescape=true,label=lst:virtHWconfig]{Name}
plant.simulated = true
plant.type = MockJAviator
plant.listener = udp://localhost:9011
plant.location.system.type = gpssim
plant.location.system.listener = tcp://localhost:9012
plant.location.system.update.rate = 10

controller.simulated = true
controller.type = JControl

pilot.type = JPilot
pilot.name = Pilot One
pilot.controller.connector = udp://localhost:9014
\end{lstlisting}
The simulated model helicopter has a simulated GPS receiver onboard, which listens on \texttt{localhost} port
\texttt{9012} for \acs{TCP} clients. It delivers the vehicle positions at a rate of 10 updates per second.  

In the configuration, shown in Listing~\ref{lst:virtHWconfig}, the
Pilot web application also emulates the controller for operating attitude and altitude. The controller uses
the property \texttt{plant.location.system.listener} to determine how to access the helicopter plant, and
uses the property \texttt{pilot.controller.connector} to identify the parameters for configuring a connection
for incoming commands.

The properties prefixed by \texttt{pilot} shows the type and name of the autopilot component, as well as
define the connection to the attitude and altitude controller.

Listing~\ref{lst:virtSensorConfig} depicts the configuration of simulated sensors. Property \texttt{sensor.list}
defines the sensors to be simulated. The names in this list prepended by \texttt{sensor} are part of the 
configuration that follows.
\lstset{tabsize=3,language=Tex}
\begin{lstlisting}[caption={Sensor Configuration Example},mathescape=true,label=lst:virtSensorConfig]{Name}
sensor.list = gps, sonar, temp, photo

sensor.gps.name = GPS receiver
sensor.gps.path = position
sensor.gps.uri = gps:///

sensor.sonar.name = Sonar
sensor.sonar.path = sonar
sensor.sonar.uri = sonar:///

sensor.temp.name = Thermometer
sensor.temp.path = temperature
sensor.temp.uri = rand:///18/22

sensor.photo.name = Belly Mounted Photo Camera
sensor.photo.path = photo
sensor.photo.uri = x11:///:21
\end{lstlisting}
Each sensor configuration defines a \texttt{name}, a \texttt{path}, and an \texttt{uri} parameter.
The parameter \texttt{name} of a sensor simply indicates its name and has no other impact whatsoever.
The parameter \texttt{path} is a suffix to an \acs{URL} dependent on the deployment context of the corresponding
Pilot web application. Let's assume, for example, the Pilot web application is deployed in a
web container's context \texttt{/pilot} listening on machine \texttt{nanook} port \texttt{9010}.
To access the Pilot web application, an operator uses the \acs{URL} \texttt{http://nanook:9010/pilot}. In this
example the \acs{URL} to access the above configured belly mounted photo camera is
\texttt{http://nanook:9010/pilot/sensor/photo}. 

Listing~\ref{lst:VclExample} illustrates a \ac{VCL} script example. 
The character \texttt{\#} indicates lines containing comments.
%
The first command in this example, ``\texttt{go auto}'', switches the vehicle into autopilot mode. Without
this command, all following commands are ignored. Command ``\texttt{takeoff}'' starts the vehicle's engines
and lifts it off the ground to an altitude of \unit{1}{\meter} within \unit{5}{\second}.
%
The ``\texttt{fly to}'' commands define waypoints the vehicle has to traverse by specifying latitude, longitude,
and altitude above ground in absolute values. Additionally, this commands define a certain precision to
determine when a waypoint has been reached, e.g., a sphere of \unit{1}{\meter} radius. Furthermore, this commands
assign an average velocity to approach waypoints, e.g., \unit{2}{\meter\per\second}.
% 
\lstset{tabsize=3,language=Tex}
\begin{lstlisting}[caption={Vehicle Control Language Example},mathescape=true,label=lst:VclExample]{Name}
##
## @(#) real vehicle set course example
##
go auto
takeoff 1m for 5s
fly to (47.82204197, 13.04086670, 20)abs precision 1m 2mps
fly to (47.82206088, 13.04092035, 20)abs precision 1m 2mps
fly to (47.82195102, 13.04488063, 20)abs precision 1m 2mps
hover for 20s
land
go manual
\end{lstlisting}
%
After all waypoints have been traversed, command ``\texttt{hover}'' instructs the vehicle to hover at
the last waypoint for, in this example, \unit{20}{\second}. After that, command ``\texttt{land}'' directs
the vehicle to land. Finally, command ``\texttt{go manual}'' switches back to manual control.

\subsection{Engine Web Application}






\subsection{Mapper Web Application}




\subsection{GM-Viewer Web Application}




\section{Section}

\subsection{Subtitle}

Plain text.

\subsection{Another subtitle}

More plain text.


\section{Mapping}
The mapper is responsible for automatically mapping virtual vehicles to real vehicles. To accomplish this, the mechanism invokes 
the migration of virtual vehicles based on a mapping decision made by a mapper algorithm. The mapper is a a servlet and additionally 
to the mapper itself with its mapping algorithms, a registration service is present. The servlet can be suspended and stores all
registration information persistent in a file, otherwise registered engines would be lost. This is important because only registered 
engines are considered during the mapping process.

\subsection{Registration Service}
An engine registers itself with the registration service upon start up using its ID. If the registration was successful, the service
fetches some useful information and stores these together with the engine id. The fetched information are sensors and 
way points (flight plan), in our case, these are static informations. If a registration attempt was not successful, the engine 
keeps trying to register until it succeeds or it is terminated.

\subsection{Mapper}
The mapper itself is cyclic. It has a working and a sleeping cycle. \\
The working cycle consists of three steps:
\begin{tabbing}
1. fetch status \= of all virtual vehicles from all registered engines \\
\>	the status message includes: \= \\ 
\> \>					the next action point \\ 
\> \>					and its actions \\[0.25cm]
2. fetch status \= of all real vehicles (on which an engine is running) \\
\>	the message includes:  \\
\> \>				the current position \\
\> \> 				the next position \\
\> \>				and the velocity \\[0.25cm]
3. execute the mapping algorithm \\
\end{tabbing}
At the time of writing, there are two algorithms available that can be set in the configuration file:
\subsubsection{Random Mapping Algorithm}
Does not use any information. Randomly selects two different engines. If one of these has a virtual vehicle, then this vehicle 
will be migrated to the other engine.

\subsubsection{Simple Mapping Algorithm}

	\begin{tabbing}
	\textbf{for} \= \textbf{all} virtual vehicles \textbf{do} \\[.25cm]
	\> \textbf{if}  \= virtual vehicle program is complete \\
	\> \>	\textbf{then} invoke migration to central engine \\[.25cm]
	\>	\textbf{else} \= find fastest real vehicle with at least one needed sensor \\
	\> \>	\textbf{and} distance CurrentToNext to ActionPoint \begin{math}< \end{math} tolerance \\[.25cm]
	\>	\textbf{if} found vehicle \textbf{then} invoke migration to it \\
	\end{tabbing}

Fastest vehicle means: the vehicle with the shortest flight time from its current position to the action point (thus: higher precision \begin{math} \rightarrow \end{math} faster)
of the virtual vehicle. 

conclusion:
A virtual vehicle stays on a real vehicle or the central engine as long as there is no suitable real vehicle, but such a vehicle can't 
be hidden by a non suitable vehicle.

%%
%% Results
%%

\chapter{Results}

This chapter summarizes the four demonstrations held in class on January 24, 2012. The main goals of this
demonstrations were to show data collecting \acp{VV} carried by \acp{RV}, as well as \acp{VV} migrating
among \acp{RV}. All images shown in this chapter were rendered by utilizing Google Maps \cite{GOOGLEMAPS2012} in a web browser.


\section{Demonstration 1: Data Collection}
In this demonstration one flying \ac{RV} carries one \ac{VV}, which collects data at four locations.
\begin{figure}[h]
	\begin{center}
		\begin{tabular}{cc}
			a)&{\includegraphics[width=9.4cm]{ese-demo1-1.png}} \\
			b)&{\includegraphics[width=9.4cm]{ese-demo1-2.png}} \\
			
			c)&{\includegraphics[width=9.4cm]{ese-demo1-3.png}} \\
			d)&{\includegraphics[width=9.4cm]{ese-demo1-4.png}} \\
			
			e)&{\includegraphics[width=9.4cm]{ese-demo1-5.png}} \\
			f)&{\includegraphics[width=9.4cm]{ese-demo1-6.png}} \\
	
			g)&{\includegraphics[width=9.4cm]{ese-demo1-7.png}} \\
			h)&{\includegraphics[width=9.4cm]{ese-demo1-8.png}}
		\end{tabular}
	\end{center}
	\caption{Data Collection Demonstration.\label{fig:demo1}}
\end{figure}
Figure~\ref{fig:demo1}~a displays \ac{RV} \textit{Pilot~One} with \ac{VV} \textit{VV~1.1} onboard.
The color green of \textit{Pilot~One} indicates that the \ac{RV} is still on the ground, and the
color yellow of \textit{VV~1.1} shows that the \ac{VV} still has action points to process.
The action points of \textit{VV~1.1} command accessing the sensors belly mounted photo camera, thermometer,
barometer, sonar, course over ground, random, GPS altitude, and speed over ground. 
In Figure~\ref{fig:demo1}~b \textit{Pilot~One} approaches the first action point. The red label indicates
that the \ac{RV} is flying. After completing the first action point \textit{Pilot~One} approaches the
second action point of \textit{VV~1.1}, as depicted in Figures~\ref{fig:demo1}~c and \ref{fig:demo1}~d.
Figures~\ref{fig:demo1}~e, \ref{fig:demo1}~f, and \ref{fig:demo1}~g visualize \textit{Pilot~One} heading
for the third and fourth action point. After processing the fourth action point, the \ac{VV}'s label
becomes green, which denotes the completion of \textit{VV~1.1}'s mission.

\section{Demonstration 2: Migration}
In this demonstration two \acp{RV} fly along their set courses and one \ac{VV} collects data at five locations.
The blue line in Figure~\ref{fig:demo2vvPath} suggests the virtual path of vehicle \textit{VV~2.1}.
\begin{figure}[h]
	\begin{center}
		{\includegraphics[width=9.4cm]{ese-demo2-1.png}}
	\end{center}
	\caption{Path of Virtual Vehicle \textit{VV~2.1}.\label{fig:demo2vvPath}}
\end{figure}
Initially, \ac{RV} \textit{Pilot~Two} carries \textit{VV~2.1}. After the \acp{RV} take off, a migration
of \textit{VV~2.1} from \textit{Pilot~Two} to \textit{Pilot~One} must take place to allow the \ac{VV} to
reach the first action point. The currently available mapping algorithm considers only the current
set course segment of a \ac{RV} for mapping decisions.
In this demonstration the first action point resides on the second segment of the  set course of
\textit{Pilot~One}. As displayed in Figure~\ref{fig:demo2mig01}, the mapper migrates \textit{VV~2.1} as
soon as \textit{Pilot~One} enters its second set course segment. 
\begin{figure}[h]
	\begin{center}
		{\includegraphics[width=9.4cm]{ese-demo2-2.png}}
	\end{center}
	\caption{First migration of \textit{VV~2.1} from \textit{Pilot~Two} to \textit{Pilot~One}
	initiated by a decision of the mapping algorithm.\label{fig:demo2mig01}}
\end{figure}
\textit{VV~2.1} resides on \textit{Pilot~One} until the \ac{RV} reaches the first action point
(Figure~\ref{fig:demo2mig02}~a). As soon as \textit{VV~2.1} has captured an image via the belly mounted camera,
the mapping algorithm decides to initiate a migration of the \ac{VV} to \textit{Pilot~Two}
(Figure~\ref{fig:demo2mig02}~b). Then, \textit{VV~2.1} takes a picture on the second action point.  
%
Now, there are no action points in the current set course segments of both \acp{RV}. In such a case
the currently implemented mapping algorithm can not decide whether migrating the \ac{VV} would be
beneficial or not. So, the \ac{VV} stays on board of \textit{Pilot~Two} until the mapping algorithm
decides otherwise.
%
\begin{figure}[h]
	\begin{center}
		\begin{tabular}{rr}
			a)&{\includegraphics[width=9.4cm]{ese-demo2-3.png}} \\
			b)&{\includegraphics[width=9.4cm]{ese-demo2-4.png}} \\
			c)&{\includegraphics[width=9.4cm]{ese-demo2-5.png}} \\
			d)&{\includegraphics[width=9.4cm]{ese-demo2-6.png}} \\
			e)&{\includegraphics[width=9.4cm]{ese-demo2-7.png}} \\
			f)&{\includegraphics[width=9.4cm]{ese-demo2-8.png}}
		\end{tabular}
	\end{center}
	\caption{\textit{Pilot~One} and \textit{Pilot~Two} mutually carrying \textit{VV~2.1}
		caused by migration decisions of the mapping algorithm.\label{fig:demo2mig02}}
\end{figure}
The \acp{RV} continue to traverse their set courses until \textit{Pilot~Two} enters the fourth segment.
The mapping algorithm detects that \textit{VV~2.1} already is onboard \textit{Pilot~Two} and suppresses
a migration. In Figures~\ref{fig:demo2mig02}~c and \ref{fig:demo2mig02}~d \textit{VV~2.1} captures
photos at the third and at the fourth action point.
%
For the fifth action point, the mapping algorithm decides a migration of \textit{VV~2.1} back to
\textit{Pilot~One}, as visualized in Figure~\ref{fig:demo2mig02}~e. After taking a picture at the last
action point, \textit{VV~2.1} has completed its mission. To indicate this, the label of \textit{VV~2.1}
turns green (Figure~\ref{fig:demo2mig02}~f).
Finally, the mapping algorithm decides a migration back to the central Engine.


\section{Demonstration 3: Different Sensors}
The \acp{RV} in this demonstration do not have the same set of sensors. The first \ac{RV}, \textit{Pilot~One},
carries a thermometer, the second \ac{RV}, \textit{Pilot~Two}, ferries a barometer, and the third \ac{RV},
\textit{Pilot~Three}, transports a belly mounted camera.
All three \acp{RV} follow the same set course in sequence and keep a flying distance of \unit{10}{\second}.
The task list of \textit{VV~3.1} consists of two action points, which require capturing photos, temperature
values, and air pressure values.   

Figure~\ref{fig:demo3mig03}~a shows all three \acp{RV} approaching the first action point. Since
\textit{Pilot~One} arrives first and provides a required thermometer sensor, the mapper algorithm already
initiated a migration of \textit{VV~3.1} to \textit{Pilot~One}. 
%
At this point in time, the indicator of the first action point visualizes all three required actions
as incomplete.
%
After \textit{VV~3.1} has completed the temperature measurement, \textit{Pilot~Two} is the next in line
to reach the action point supplying a fitting sensor. Therefore, the mapping algorithm commands a migration 
to \textit{Pilot~Two}.
%
As shown in Figure~\ref{fig:demo3mig03}~b, the action point indicator no more views
the thermometer.
\begin{figure}[h]
	\begin{center}
		\begin{tabular}{rr}
			a)&{\includegraphics[width=9.4cm]{ese-demo3-1.png}} \\
			b)&{\includegraphics[width=9.4cm]{ese-demo3-2.png}} \\
			c)&{\includegraphics[width=9.4cm]{ese-demo3-3.png}} \\
			d)&{\includegraphics[width=9.4cm]{ese-demo3-4.png}}
		\end{tabular}
	\end{center}
	\caption{\acp{RV} \textit{Pilot~One}, \textit{Pilot~Two}, and \textit{Pilot~Three} mutually carrying
	 	\textit{VV~3.1} repeatedly to the first action point caused by migration decisions of the mapping
	 algorithm.\label{fig:demo3mig03}}
\end{figure}

Figure~\ref{fig:demo3mig03}~c presents the situation after \textit{VV~3.1} has measured the air pressure.
The mapper algorithm already ordered a migration of \textit{VV~3.1} to \textit{Pilot~Three} and
the action point indicator views the remaining action for taking a photo. 
%
Then, \textit{VV~3.1} stays onboard of \textit{Pilot~Three}, because the mapping algorithm can not find
a eligible \ac{RV} for processing the next action point. 

In Figure~\ref{fig:demo3mig04}~a \textit{Pilot~One} enters a set course segment that leads to the next
action point. Since \textit{Pilot~One} facilitates an adequate sensor and is the only one, as the mapping algorithm
considers it, to reach the action point, the mapper algorithm directs a migration of \textit{VV~3.1} to
\textit{Pilot~One}.

As \textit{Pilot~One} catches the action point, \textit{VV~3.1} queries the thermometer and the mapper
algorithm orders a migration to \textit{Pilot~Two} (Figure~\ref{fig:demo3mig04}~b).
Again, the action point indicator no more views the thermometer.

\begin{figure}[h]
	\begin{center}
		\begin{tabular}{rr}
			a)&{\includegraphics[width=9.4cm]{ese-demo3-5.png}} \\
			b)&{\includegraphics[width=9.4cm]{ese-demo3-6.png}} \\
			c)&{\includegraphics[width=9.4cm]{ese-demo3-7.png}} \\
			d)&{\includegraphics[width=9.4cm]{ese-demo3-8.png}}
		\end{tabular}
	\end{center}
	\caption{\acp{RV} \textit{Pilot~One}, \textit{Pilot~Two}, and \textit{Pilot~Three} mutually carrying
		\textit{VV~3.1} repeatedly to the second action point caused by migration decisions of the mapping
		algorithm.\label{fig:demo3mig04}}
\end{figure}
After that, \textit{Pilot~Two}
arrives at the action point, \textit{VV~3.1} measures the air pressure and the mapper algorithm commands
a migration to \textit{Pilot~Three} (Figure~\ref{fig:demo3mig04}~c).
Consequently, the air pressure symbol vanishes from the action point indicator.
%
Now, \textit{Pilot~Three} gets to the action point and \textit{VV~3.1} completes its mission by 
taking a picture, as shown in Figure~\ref{fig:demo3mig04}~d. The \ac{VV}'s label turns green to show this.
%
At last, the mapping algorithm initiates a migration back to the central Engine.


\section{Demonstration 4: Multiple Virtual Vehicles}
In this demonstration three \acp{RV} fly along their set courses and four \acp{VV} collect data at several locations.
Each \ac{RV} provides the same set of sensors. Initially, the \acp{VV} idle on the central Engine and wait for
the mapping algorithm to assign an eligible \ac{RV}.  
%
The blue lines in Figure~\ref{fig:demo4img1}~a display the virtual paths of the \acp{VV}.
All \ac{VV} paths progress top-down, as indicated by black arrows.

Figure~\ref{fig:demo4img1}~b shows an advanced stage of this demonstration mission displaying all four \acp{VV} in
action.
% \begin{figure}[h]
% 	\begin{center}
% 		a) {\includegraphics[width=11.1cm]{ese-demo4-1.png}}
% 	\end{center}
% 		\caption{Demonstration 4: Virtual Vehicle Paths.\label{fig:demo4img1}}
% \end{figure}
%
\begin{figure}[h]
	\begin{center}
		\begin{tabular}{rr}
		a)&{\includegraphics[width=11.42cm]{ese-demo4-1.png}} \\
		b)&{\includegraphics[width=11.42cm]{ese-demo4-2.png}}
		\end{tabular}
	\end{center}
		\caption{Demonstration 4: a) Virtual Vehicle Paths. 
		b) Multiple \acsp{VV} in action.\label{fig:demo4img1}}
\end{figure}


%%
%% Conclusion
%%

\chapter{Conclusion}

This work has presented an implementation of a simulation system that demonstrates information-acquisition-as-a-service
of mobile sensor networks for \ac{CPCC} as proposed in \cite{HotCloud10}.
%
This chapter concludes the paper by summarizing the current situation, and providing suggestions for future
enhancements.

Our goal was to implement a flexible and scalable simulation system. We applied 
the Java programming language and standard Internet technologies like web services to meet this challenge.  

The implemented system currently allows the simulation of helicopter fleets of several dozens of vehicles
and supports the simulation of sensors like GPS receivers and photo cameras.
To simulate air-pressure sensors, temperature sensors, etc.~the system utilizes random number generators, which
deliver values in a defined range and precision.
%
Support for hardware-in-the-loop testing is available for flight control systems and helicopter plants.
%

Simulated helicopters follow strict flight plans, but do not access the onboard sensors for data collection.
It is a virtual abstraction of autonomous vehicles, \acfp{VV} for short, that gathers data.
One helicopter is able to carry several \acp{VV}. To complete their missions, \acp{VV} may migrate between helicopters.


Future works could cover the following topics:
\begin{itemize}
  \item The implemented mapping algorithm considers only the current flight plan segment for migration decisions.
  	Future implementations should include all helicopter set course segments.
   
   \item Flight plans for helicopters should be derived from \ac{VV} mission requirements. 
   
   \item Although the implementation is able to simulate dozens of helicopters, the network traffic between
   	helicopters and ground station needs optimization to achieve higher scalability.
   
   \item More advanced camera sensors may allow for directing the sensors towards defined targets, which requires
   	extending the \ac{VV} programming language.
   
   \item In \ac{VV} missions action points define where to capture sensor values. Video cameras  
   	and other streaming sources need new \ac{VV} programming language commands to trigger recordings.

\end{itemize}





\bibliographystyle{references}
\bibliography{references}
\vfill\newpage

\chapter*{List of Abbreviations\markboth{List of abbreviations}{List of abbreviations}}
\addcontentsline{toc}{chapter}{List of abbreviations}
\thispagestyle{plain}
%%
%% Abbreviations
%%

\begin{acronym}
	\acro{RV}{Real Vehicle}
	\acro{VV}{Virtual Vehicle}
	\acro{UAV}{Unmanned Aerial Vehicle}
	\acro{CPCC}{cyber-physical cloud computing}
\end{acronym}



\vfill\newpage


\end{document}
